%% LyX 1.6.5 created this file.  For more info, see http://www.lyx.org/.
%% Do not edit unless you really know what you are doing.
\documentclass[english]{article}
\usepackage[T1]{fontenc}
\usepackage[latin9]{inputenc}
\usepackage[a4paper]{geometry}
\geometry{verbose,tmargin=3cm,bmargin=3cm,lmargin=2cm,rmargin=2cm}
\setlength{\parskip}{\medskipamount}
\setlength{\parindent}{0pt}

\makeatletter
%%%%%%%%%%%%%%%%%%%%%%%%%%%%%% User specified LaTeX commands.
\usepackage{fancyhdr}
\pagestyle{fancy}
\rhead{Notes\ on\ pricing\ of\ crop\ monitoring\ data,\ \today}

\makeatother

\usepackage{babel}

\begin{document}

\section*{The problem}

We want to monitor the spread of viral disease among staple crops.
Interested in two things: incidence and severity.

Survey workers can provide geotagged images of crops, in return for
micropayments of airtime or mobile money.

Use changing prices (as a function of location) to create the right
bias for data collection.

Assume computer vision software on the survey device can reject images
which have low information (not of a leaf/out of focus).


\section*{The model}

Start with a Gaussian process with squared exponential covariance
(lat,lon,time). 

Advantages: posterior is centred around the data points. Disadvantages:
could become computationally expensive with large numbers of datapoints.

Two coupled models: severity and incidence. (Are these two things
related? Or conditionally independent given the observations?)

Nature of observations: tropical agriculture standards for levels
of disease severity, CMD symptoms are quantified from 0 (no damage)
to 5 (very damaged). Seems easiest to use these; a lot of thought
has gone into this quantification. Alternatives: degree of belief
that plant is infected, problematic because this tops out.

Underlying severity: use the data space directly (range 0-5), or use
range $+\infty$ to$-\infty$ and squash through a logistic sigmoid?

Incidence is more complicated: this is a partially observed point
process. The underlying state could either be something like a Poisson
rate (probability per unit area that there is an infected plant),
or a probability that a given plant is infected.

We get a posterior $\mu,S|x_{*},x,y$ . The covariance $S$ is used
in the pricing.

When using a temporal model, to make this stable we need to work out
which points past in time to {}``forget''.


\section*{Pricing data collection}

We want images from the places where our uncertainty is highest (about
either severity or incidence). Therefore we price according to uncertainty.
Active learning.

We are more interested in cases of disease than non-disease (how can
this be quantified?). 

We are more interested in areas where there is high cassava cultivation.

If GP mean is zero, then non-zero observations will have high surprise.

Price is therefore at the minimum a function of two covariances (incidence/severity),
cassava cultivation density.


\section*{Practical issues}

How often should the pricing be updated? If someone takes a picture
of a plant, we don't want too many more pictures of that plant. We
want that person to move away. Ideally we would recalculate the pricing
there and then. But person may not have network to communicate there
and then. Would also have to communicate the updated map to everyone
using the sytem. Possibly download some global pricing map each day,
then local computations adjust the price. This doesn't stop several
people photographing the same plant though and expecting the same
price.

Creating a market might mean the survey becomes autonomous; people
buy location aware survey phones with microcredit, then make earnings
to pay off the loans.
\end{document}
